%---------- Inleiding ---------------------------------------------------------

\section{Introductie}%
\label{sec:introductie}

LaTeX (\LaTeX) uitspraak: la-tech; is een populair softwaresysteem in de wetenschappelijke wereld omdat het uitblinkt in het zetten van technische documenten,
en beschikbaar is voor bijna alle computersystemen \autocite{Wiki23}.

Niet alleen in de wereld van de wetenschap wordt LaTeX gebruikt. Ook studenten op hoge scholen en universiteiten maken er gebruik van voor het schrijven van bachelor- en/of masterproeven.

Bij het schrijven van, al dan niet wetenschappelijke, teksten is het uitermate belangrijk om aan een correcte vorm van bronvermelding te doen.

Binnen LaTeX zijn er verschillende manieren om dit aan te pakken. Eén van deze manieren is met behulp van BibLaTeX, een package speciaal gebouwd voor deze taak.

Studenten te HoGent dienen gebruik te maken van deze combinatie bij het schrijven van hun bachelorproef. Ondanks dat lectoren veel moeite steken in het bondig toelichten van het correcte gebruik, worden er nog veel fouten gemaakt op het correct bijhouden van bronnen. 
Op deze groep zal deze bachelorproef zich focussen, met behulp van een statische analysetool, ook wel linter genoemd, zouden er al veel van de herhalende fouten voorkomen kunnen worden. Een linter is een programma dat broncode of gestructureerde dataformaten kan controleren op stijl, syntax en logische fouten \autocite{Kamunya2023}.

Het zou dus uitermate geschikt zijn om de studenten een linter te laten gebruiken om hen zo te helpen bij het voorkomen of opsporen van de gemaakte fouten. Zo dienen lectoren hen niet keer op keer op dezelfde fouten te wijzen.

BibLaTeX is voortkomende uit BibTex en biedt meer opties om bibliografieën en citaten te configureren \autocite{Cassidy2013}. Hoewel er voor BibTex reeds een linter bestaat, is deze niet compatibel met BibLaTeX.

Het doel van deze bachelorproef is om een proof-of-concept, analyse \& de software-architectuur uit te werken voor een BibLaTeX linter en er een prototype voor te schrijven in een passende programmeertaal. 

Concreet betekent dit:
\begin{itemize}
  \item De lijst van gewenste functionele en niet-functionele requirements aanvullen en structureren naar prioriteiten
  \item De werking van bestaande linters bestuderen als inspiratiebron
  \item Een gemotiveerde keuze maken voor de te gebruiken programmeertaal en eventuele libraries
  \item Een prototype met een minimale set van linting-regels implementeren
  \item Unit tests schrijven met zo compleet mogelijke code coverage
  \item CI pipeline opzetten voor packaging en testing
  \item Documentatie schrijven voor het gebruik en uitbreiden van de linting-regels
\end{itemize}

Als eindresultaat voor deze bachelorproef zal er een open-source prototype opgesteld worden waaraan vrijwilligers verder kunnen werken.


%---------- Stand van zaken ---------------------------------------------------

\section{State-of-the-art}%
\label{sec:state-of-the-art}

Op het ogenblik van het schrijven van dit bachelorproefvoorstel, zijn er nog geen optimale BibLaTeX-Linters beschikbaar. De enige beschikbare linter die bestaat voor BibLaTeX voor dit moment, staat op een GitHub-repository van 
Pez Cuckow \autocite{Cuckow2022}. 


----

Concreet betekent dit:
\begin{itemize}
  \item De lijst van gewenste functionele en niet-functionele requirements aanvullen en structureren naar prioriteiten
  \item De werking van bestaande linters bestuderen als inspiratiebron
  \item Een gemotiveerde keuze maken voor de te gebruiken programmeertaal en eventuele libraries
  \item Een prototype met een minimale set van linting-regels implementeren
  \item Unit tests schrijven met zo compleet mogelijke code coverage
  \item CI pipeline opzetten voor packaging en testing
  \item Documentatie schrijven voor het gebruik en uitbreiden van de linting-regels
\end{itemize}

Als eindresultaat voor deze bachelorproef zal er een open-source prototype opgesteld worden waaraan vrijwilligers verder kunnen werken.


%---------- Methodologie ------------------------------------------------------
\section{Methodologie}%
\label{sec:methodologie}

Hier beschrijf je hoe je van plan bent het onderzoek te voeren. Welke onderzoekstechniek ga je toepassen om elk van je onderzoeksvragen te beantwoorden? Gebruik je hiervoor literatuurstudie, interviews met belanghebbenden (bv.~voor requirements-analyse), experimenten, simulaties, vergelijkende studie, risico-analyse, PoC, \ldots?

Valt je onderwerp onder één van de typische soorten bachelorproeven die besproken zijn in de lessen Research Methods (bv.\ vergelijkende studie of risico-analyse)? Zorg er dan ook voor dat we duidelijk de verschillende stappen terug vinden die we verwachten in dit soort onderzoek!

Vermijd onderzoekstechnieken die geen objectieve, meetbare resultaten kunnen opleveren. Enquêtes, bijvoorbeeld, zijn voor een bachelorproef informatica meestal \textbf{niet geschikt}. De antwoorden zijn eerder meningen dan feiten en in de praktijk blijkt het ook bijzonder moeilijk om voldoende respondenten te vinden. Studenten die een enquête willen voeren, hebben meestal ook geen goede definitie van de populatie, waardoor ook niet kan aangetoond worden dat eventuele resultaten representatief zijn.

Uit dit onderdeel moet duidelijk naar voor komen dat je bachelorproef ook technisch voldoen\-de diepgang zal bevatten. Het zou niet kloppen als een bachelorproef informatica ook door bv.\ een student marketing zou kunnen uitgevoerd worden.

Je beschrijft ook al welke tools (hardware, software, diensten, \ldots) je denkt hiervoor te gebruiken of te ontwikkelen.

Probeer ook een tijdschatting te maken. Hoe lang zal je met elke fase van je onderzoek bezig zijn en wat zijn de concrete \emph{deliverables} in elke fase?

%---------- Verwachte resultaten ----------------------------------------------
\section{Verwacht resultaat, conclusie}%
\label{sec:verwachte_resultaten}

Hier beschrijf je welke resultaten je verwacht. Als je metingen en simulaties uitvoert, kan je hier al mock-ups maken van de grafieken samen met de verwachte conclusies. Benoem zeker al je assen en de onderdelen van de grafiek die je gaat gebruiken. Dit zorgt ervoor dat je concreet weet welk soort data je moet verzamelen en hoe je die moet meten.

Wat heeft de doelgroep van je onderzoek aan het resultaat? Op welke manier zorgt jouw bachelorproef voor een meerwaarde?

Hier beschrijf je wat je verwacht uit je onderzoek, met de motivatie waarom. Het is \textbf{niet} erg indien uit je onderzoek andere resultaten en conclusies vloeien dan dat je hier beschrijft: het is dan juist interessant om te onderzoeken waarom jouw hypothesen niet overeenkomen met de resultaten.

