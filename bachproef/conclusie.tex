%%=============================================================================
%% Conclusie
%%=============================================================================

\chapter{Conclusie}%
\label{ch:conclusie}

% TODO: Trek een duidelijke conclusie, in de vorm van een antwoord op de
% onderzoeksvra(a)g(en). Wat was jouw bijdrage aan het onderzoeksdomein en
% hoe biedt dit meerwaarde aan het vakgebied/doelgroep? 
% Reflecteer kritisch over het resultaat. In Engelse teksten wordt deze sectie
% ``Discussion'' genoemd. Had je deze uitkomst verwacht? Zijn er zaken die nog
% niet duidelijk zijn?
% Heeft het onderzoek geleid tot nieuwe vragen die uitnodigen tot verder 
%onderzoek?

% Wat zijn de requirements die in de POC zijn voldaan en welke dienen nog verricht te worden? 

% Conclusie, huidige stand, toekomst visie?


Samenvattend biedt bibla een meer uitgebreide en gedetailleerde set regels die beter zijn afgestemd op de eisen van BibLaTeX in vergelijking met bibl. De aanvullende en aangepaste regels zorgen voor een grotere nauwkeurigheid en consistentie in de verwerking van bibliografische vermeldingen. Door het volgen van deze regels kunnen gebruikers ervoor zorgen dat hun BibLaTeX-bestanden voldoen aan de huidige standaarden en best practices die worden opgelegd door HOGENT.