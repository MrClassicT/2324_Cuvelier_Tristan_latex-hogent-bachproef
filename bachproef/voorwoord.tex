%%=============================================================================
%% Voorwoord
%%=============================================================================

\chapter*{\IfLanguageName{dutch}{Woord vooraf}{Preface}}%
\label{ch:voorwoord}

%% TODO:
%% Het voorwoord is het enige deel van de bachelorproef waar je vanuit je
%% eigen standpunt (``ik-vorm'') mag schrijven. Je kan hier bv. motiveren
%% waarom jij het onderwerp wil bespreken.
%% Vergeet ook niet te bedanken wie je geholpen/gesteund/... heeft

Het schrijven van deze bachelorproef was een uitdagende en leerzame ervaring die ik niet had kunnen voltooien zonder de steun en begeleiding van verschillende mensen. Allereerst wil ik mijn promotor en co-promotor, \texttt{Dhr. Bert Van Vreckem}, bedanken voor zijn voortdurende steun, geduld en waardevolle feedback gedurende het hele proces. Zijn deskundigheid en begeleiding hebben mijn onderzoek naar een hoger niveau getild en dit project mogelijk gemaakt. Ik had geen geschiktere persoon kunnen wensen om me hierbij te begeleiden.
\\ \newline{}
Daarnaast wil ik mijn dank uitspreken naar mijn docenten van het Departement IT en Digitale Innovatie, die mij de benodigde kennis en vaardigheden hebben bijgebracht gedurende mijn opleiding. Ook wil ik mijn medestudenten bedanken voor hun steun en samenwerking. Hun feedback was meer dan welkom en zonder hen zouden er nog meer bugs in de linter zitten en zou het nut ervan nog niet bewezen zijn. In het bijzonder wil ik mijn medestagiairs, \texttt{Jasper V.D.} en \texttt{Sarah E.}, extra bedanken voor de last-minute bug reports die hebben geholpen om meerdere toekomstige problemen te voorkomen.
\\ \newline{}
Ik heb gekozen voor het onderwerp \texttt{"Proof Of Concept: Linter voor BibLaTeX"} vanwege mijn interesse in softwareontwikkeling en mijn passie voor het helpen van mensen. Het ontwikkelen van een linter voor BibLaTeX biedt een concrete oplossing voor een veelvoorkomend probleem bij studenten en onderzoekers, namelijk het correct beheren van bibliografische referenties. Door dit probleem aan te pakken, kon ik een impact maken op de kwaliteit van bachelorproeven en andere (academische) werken, waardoor ik wellicht meer mensen heb kunnen helpen dan ik me kon voorstellen.
\\ \newline{}
Deze bachelorproef onderzoekt de huidige stand van zaken op het gebied van BibLaTeX-linters, de technische uitdagingen en mogelijkheden, en presenteert een proof of concept voor een linter die specifiek is ontworpen voor gebruik met BibLaTeX. Het doel is om een hulpmiddel te bieden dat fouten in bibliografische referenties automatisch detecteert en corrigeert, wat de kwaliteit van academisch werk ten goede komt.
\\ \newline{}
De afgelopen maanden waren intensief maar ook verrijkend. Ik heb veel geleerd over softwareontwikkeling, projectmanagement en het uitvoeren van technisch onderzoek. Dit project heeft mijn interesse in het vakgebied verder aangewakkerd en ik kijk ernaar uit om de groei van de linter te kunnen volgen.
\\ \newline{}
Tot slot wens ik graag een woord van dank te richten aan \texttt{Arne Van Den Kerchove} voor het maken van bibl, wat uiteindelijk de hele basis werd van deze proof of concept. Zonder zijn professionele werk zou bibla niet staan waar het vandaag staat. Ook wil ik \texttt{Alexander Veldeman} bedanken, die als eerste de kans greep om de online versie van bibla te testen en mij extra energie gaf met zijn enthousiasme over dit project. Daarnaast wil ik mijn vriendin bedanken, omdat ze voor mij gezorgd heeft tijdens deze drukke periodes en steeds met veel geduld bleef omgaan met mij. Ik ben er mij van bewust dat het niet altijd even leuk was.
\\ \newline{}
Ik hoop dat dit werk bijdraagt aan verdere ontwikkelingen op dit gebied en dat het nuttig zal zijn voor toekomstige studenten en onderzoekers.
\\ \newline{}
Met dank en waardering,
\\ \newline{}
Tristan Cuvelier
