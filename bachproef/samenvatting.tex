%%=============================================================================
%% Samenvatting
%%=============================================================================

% TODO: De "abstract" of samenvatting is een kernachtige (~ 1 blz. voor een
% thesis) synthese van het document.
%
% Een goede abstract biedt een kernachtig antwoord op volgende vragen:
%
% 1. Waarover gaat de bachelorproef?
% 2. Waarom heb je er over geschreven?
% 3. Hoe heb je het onderzoek uitgevoerd?
% 4. Wat waren de resultaten? Wat blijkt uit je onderzoek?
% 5. Wat betekenen je resultaten? Wat is de relevantie voor het werkveld?
%
% Daarom bestaat een abstract uit volgende componenten:
%
% - inleiding + kaderen thema
% - probleemstelling
% - (centrale) onderzoeksvraag
% - onderzoeksdoelstelling
% - methodologie
% - resultaten (beperk tot de belangrijkste, relevant voor de onderzoeksvraag)
% - conclusies, aanbevelingen, beperkingen
%
% LET OP! Een samenvatting is GEEN voorwoord!

%%---------- Nederlandse samenvatting -----------------------------------------
%
% TODO: Als je je bachelorproef in het Engels schrijft, moet je eerst een
% Nederlandse samenvatting invoegen. Haal daarvoor onderstaande code uit
% commentaar.
% Wie zijn bachelorproef in het Nederlands schrijft, kan dit negeren, de inhoud
% wordt niet in het document ingevoegd.

\IfLanguageName{english}{%
\selectlanguage{dutch}
\chapter*{Samenvatting}
\selectlanguage{english}
}{}

%%---------- Samenvatting -----------------------------------------------------
% De samenvatting in de hoofdtaal van het document

\chapter*{\IfLanguageName{dutch}{Samenvatting}{Abstract}}

Deze bachelorproef richt zich op de ontwikkeling van een \acrfull{POC} voor een BibLaTeX-linter. Het doel van dit project is om een open-source tool te creëren die studenten en onderzoekers helpt bij het identificeren en bewustmaken van fouten in bibliografische referenties in BibLaTeX.

De motivatie voor deze studie komt voort uit de behoefte aan een betrouwbaar hulpmiddel om veelvoorkomende fouten in referentielijsten te detecteren. Meer bepaald door lectoren aan HOGENT, de \acrlong{POC} volgt dan ook hun ideologie. Dit helpt de kwaliteit en nauwkeurigheid van academische werken te verbeteren. Daarnaast hoeven lectoren aan HOGENT niet telkens dezelfde fouten aan te duiden bij studenten, aangezien het vaak dezelfde fouten zijn die terugkeren. Hierdoor komt er tijd vrij om belangrijkere fouten op te merken en kan het niveau naar een hoger niveau worden getild.

Het onderzoek is uitgevoerd door middel van een uitgebreide literatuurstudie, een technische analyse van bestaande linters en programmeertalen, en de eigenlijke softwareontwikkeling van de linter. De literatuurstudie gaf inzicht in de huidige stand van zaken en best practices, terwijl de technische analyse zich richtte op het evalueren van geschikte programmeertalen en tools. De softwareontwikkeling omvatte het ontwerpen, implementeren en testen van de linter, met aandacht voor gebruiksvriendelijkheid en uitbreidbaarheid.

De resultaten van dit project tonen aan dat de ontwikkelde linter effectief is in het detecteren van fouten in BibLaTeX-referenties. De linter werd grondig getest door verschillende gebruikers, wat leidde tot waardevolle feedback en suggesties voor verdere verbeteringen. De belangrijkste bijdrage van dit onderzoek is een solide basis voor een linter die verder ontwikkeld en aangepast kan worden door de community en bewezen is om compatibel te zijn met BibLaTeX.

De relevantie van deze resultaten voor het werkveld is aanzienlijk. De linter kan door studenten, docenten, onderzoekers en \LaTeX-gebruikers in het algemeen worden gebruikt om de kwaliteit van hun academische werk te verhogen, wat de betrouwbaarheid en professionaliteit van hun publicaties ten goede komt. Bovendien biedt het project een platform voor toekomstige ontwikkelingen en verbeteringen binnen de open-source community.

Met deze Proof of Concept is een belangrijke stap gezet naar een volledig functionele BibLaTeX-linter. Het is de hoop dat dit project als fundament zal dienen voor verdere innovatie en samenwerking in het academische veld.

