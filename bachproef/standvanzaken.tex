\chapter{\IfLanguageName{dutch}{Stand van zaken}{State of the art}}%
\label{ch:stand-van-zaken}

% Tip: Begin elk hoofdstuk met een paragraaf inleiding die beschrijft hoe dit hoofdstuk past binnen het geheel van de bachelorproef. Geef in het bijzonder aan wat de link is met
% het vorige en volgende hoofdstuk.

% Pas na deze inleidende paragraaf komt de eerste sectiehoofding.

% Dit hoofdstuk bevat je literatuurstudie. De inhoud gaat verder op de inleiding, maar zal het onderwerp van de bachelorproef *diepgaand* uitspitten. De bedoeling is dat de lezer na
% lezing van dit hoofdstuk helemaal op de hoogte is van de huidige stand van zaken (state-of-the-art) in het onderzoeksdomein. Iemand die niet vertrouwd is met het onderwerp, weet nu
% voldoende om de rest van het verhaal te kunnen volgen, zonder dat die er nog andere informatie moet over opzoeken \autocite{Pollefliet2011}.

% Je verwijst bij elke bewering die je doet, vakterm die je introduceert, enz.\ naar je bronnen. In \LaTeX{} kan dat met het commando \texttt{$\backslash${textcite\{\}}} of
% \texttt{$\backslash${autocite\{\}}}. Als argument van het commando geef je de ``sleutel'' van een ``record'' in een bibliografische databank in het Bib\LaTeX{}-formaat (een
% tekstbestand). Als je expliciet naar de auteur verwijst in de zin (narratieve referentie), gebruik je \texttt{$\backslash${}textcite\{\}}. Soms is de auteursnaam niet expliciet een
% onderdeel van de zin, dan gebruik je \texttt{$\backslash${}autocite\{\}} (referentie tussen haakjes). Dit gebruik je bv.~bij een citaat, of om in het bijschrift van een overgenomen
% afbeelding, broncode, tabel, enz. te verwijzen naar de bron. In de volgende paragraaf een voorbeeld van elk.

% \textcite{Knuth1998} schreef een van de standaardwerken over sorteer- en zoekalgoritmen. Experten zijn het erover eens dat cloud computing een interessante opportuniteit vormen,
% zowel voor gebruikers als voor dienstverleners op vlak van informatietechnologie~\autocite{Creeger2009}.

% Let er ook op: het \texttt{cite}-commando voor de punt, dus binnen de zin. Je verwijst meteen naar een bron in de eerste zin die erop gebaseerd is, dus niet pas op het einde van
% een paragraaf.
% ----------------------------------------------------------------------------------------------------------------------------------------------------------------------------------------------------------------------------------
Dit hoofdstuk licht toe wat een linter is en wat LaTeX(\LaTeX) en BibLaTeX zijn. Daarnaast wordt er meer verteld over de huidige stand van zaken binnen de wereld van BibLaTeX-linters, het toont aan waarom het gepast is om deze proof of concept uit te werken en de opportuniteiten die zich bieden binnen dit onderwerp. Alsook wordt er een diepere kijk gegeven aan andere linters hun functionaliteiten en performantie om een beter inzicht te verwerven in aspecten die van belang kunnen zijn voor het uitwerken van een eigen linter.

\section{Wat is een linter?}
Alvorens er uitgelegd wordt waarom het nuttig is om deze proof of concept uit te werken, is het belangrijk om een goed begrip te hebben van wat er effectief gemaakt wordt en wat het het nut ervan is. Het doel van deze proof of concept is om een linter te maken. Maar wat is dat juist?

Linten verwijst naar het proces van broncode automatisch te controleren op programmatische en stilistische fouten. Een linter scant programmatisch je code om te controleren of er problemen zijn die kunnen leiden tot bugs of inconsistenties met de code-stijl en -gezondheid. De linter is de tool die gebruikt wordt om de broncode te linten. \autocite{Kamunya2023}

Een linter is dus een statische analysetool, dat broncode of andere gestructureerde data kan analyseren.

\section{\LaTeX}
\LaTeX (uitspraak: LaTech), een uitbreiding op het Tex-typesetting systeem van Donald E. \textcite{Knuth1984}, is bedoeld voor het opmaken van tekst en wiskundige formules. TeX werd ontwikkeld in 1977 met als doel de typografische kwaliteit te verbeteren. De stabiele versie van TeX kwam uit in 1982 en ondersteunt meerdere talen en 8-bit karakters. LaTeX zelf, ontwikkeld door Leslie \textcite{Lamport1994} in 1985, voegt een reeks macro's toe om het gebruik van TeX te vereenvoudigen, en heeft zich ontwikkeld tot een standaard voor het produceren van wetenschappelijke en wiskundige documentatie. \autocite{Oetiker2023}

\subsection{Werking}
LaTeX functioneert door middel van commando's die de logische structuur van een document definiëren (zoals hoofdstukken, secties, en paragrafen). Dit is anders dan WYSIWYG (What You See Is What You Get) tekstverwerkers zoals Microsoft Office Word, waar de lay-out interactief wordt bepaald tijdens het typen. LaTeX vereist dat de auteur zijn tekst structureert met behulp van vooraf gedefinieerde commando's die de inleiding van het document bepalen. \autocite{Oetiker2023}

\subsection{Voordelen}
\begin{enumerate}
    \item Het biedt professionele vormgeving van lay-outs.
    \item Het ondersteunt de zetting van wiskundige formules.
    \item Het moedigt aan tot goed gestructureerd schrijven. Dit resulteert in duidelijk georganiseerde documenten.
    \item Er zijn veel uitbreidingen beschikbaar via packages om functionaliteit zoals PDF-output en betere font-ondersteuning toe te voegen.
\end{enumerate}

\subsection{Nadelen}
\begin{enumerate}
    \item Het instellen van een volledig nieuwe lay-out kan ingewikkeld en tijdrovend zijn.
    \item Niet ideaal voor zeer ongestructureerde documenten.
\end{enumerate}

\subsection{Conclusie}
LaTeX wordt vooral gewaardeerd in academische en technische kringen waar de precisie van de inhoud en de structuur voorop staan. Het systeem blijft populair vanwege zijn stabiliteit, de brede ondersteuning van wiskundige formules, en de mogelijkheid om complexe documenten zoals proefschriften en wetenschappelijke artikelen nauwkeurig op te maken, ook wel typesetten genoemd. \autocite{Oetiker2023}

\section{Huidige stand van zaken}
Op het ogenblik van het schrijven, zijn er nog geen \emph{optimale} BibLaTeX-Linters beschikbaar en de linters van de voorganger BibTex zijn niet compatibel met BibLaTeX. De enige beschikbare linter, die bestaat voor BibLaTeX voor dit moment, staat op een GitHub-repository van Pez Cuckow. Deze hoort functioneel te zijn, maar lijkt niet optimaal wat de code betreft. Er is dus duidelijk mogelijkheid tot verbetering. De BibLaTeX-Linter van Pez Cuckow is geschreven in Python en heeft een webinterface.\autocite{Cuckow2022} Naast het feit dat deze lastig werkend te krijgen was, werkt deze ook zeker niet zonder fouten. Het is mogelijk om deze effectief uit te testen, maar bij het testen werd er snel gemerkt dat er fouten optraden die niet snel op te lossen waren. Het was dus niet mogelijk om zomaar elk .bib bestand te gebruiken bij deze checker, wat het niet gunstig maakt om te gebruiken. Wel was het interessant om te kijken hoe Cuckow bepaalde aspecten interpreteerde en uitvoerde. Dit was uiteindelijk ook een bron van waaruit inspiratie gehaald kon worden om zo verder uit te werken.

\section{Andere linters}
Gezien het wat BibLaTeX linters betreft zeer beperkt is, werd er op zoek gegaan naar andere soorten linters. Bijvoorbeeld een linter voor JavaScript, Python of andere programmeertalen. Er werd gekeken naar het soort features dat de linters aanbieden, hun performantie, alsook als hun broncode indien deze toegankelijk was. Enkele linters die bekeken werden zijn: JSHint, Stylelint, Ruff, PyType en BibL.

Daaruit bleek dat naast de manier van implementeren, de taal waaruit de linter opgebouwd is ook van belang is voor de performantie. Dit was het begin naar een onderzoek voor de ideale programmeertaal te vinden voor deze proof of concept. 

\subsection{Linters onder de loep}
\subsubsection{Ruff}
Ruff is een Python linter die door \textcite{Astral2024} ontwikkeld is. In vergelijking met andere Python linters is Ruff veel performanter. Dit is omdat Ruff in tegenstelling tot de concurrenten, geschreven is in Rust in plaats van Python. Op de website van Astral, makers van Ruff, is er een vergelijking te zien tussen Ruff en andere Python linters die wel in Python geschreven zijn. Astral is niet de enige die dit aantoont, ook \textcite{TurnerTrauring2023} schrijft in een artikel dat de snelheid van Ruff zeker merkbaar is en toont dit aan met behulp van uitgevoerde tests. Vooral bij pipelines op virtuele machines gezien deze vCPU's (virtuele processor units) gebruiken en deze meestal trager zijn dan de processing units die terug te vinden zijn in moderne laptops of desktops.

\subsubsection{BibL}
BibL\footnote{\url{https://gitlab.com/arnevdk/bibl}} is een linter voor BibTex, de voorloper van BibLaTeX en is geschreven in Python. Het kijken naar een linter van de voorloper, leek zeer interessant om een goed idee te krijgen van wat er exact verwacht zou kunnen worden bij een linter. Zo is er een hele lijst van regels te zien alsook de projectstructuur en implementatiewijze van zowel lintercode als de testen die ervoor bestaan. Dit zal ongetwijfeld een grote inspiratie zijn voor de effectieve uitwerking.
% --- continue reading/working from here. ---
\section{Programmeertaal}
Wat de programmeertaal voor de proof of concept betreft, wordt er zich beperkt tot Rust, Python en JavaScript. Er zal een beperkt prototype uitgewerkt worden in elk van deze talen en deze zullen met elkaar vergeleken worden om te bepalen welke taal het meest geschikt is. Alle opties zijn cross-platform, wat belangrijk is gezien de linter bruikbaar dient te zijn voor iedereen. Alsook dient het een programmeertaal te zijn die gekend is.

\subsection{Rust}
Gezien de performantie van Ruff, lijkt Rust een gepaste optie om de linter in te maken. Ook de stijgende populariteit van de taal maakt dit een aantrekkelijke optie. Ondanks dat dit een onbekende taal is, lijkt het zeker wel een interessante taal om te leren. Echter komt de haalbaarheid van de proof of concept meer in gevaar door deze keuze. Hoewel de populariteit stijgt, blijft het wel nog een taal die niet door iedereen gekend is, in tegenstelling tot bijvoorbeeld JavaScript of Python. Een ander groot pluspunt aan Rust is dat code die vandaag compiled, ook nog binnen 5 of 10 jaar zal kunnen compilen, dus omdat ze streven voor een super compatibele programmeertaal te zijn. Wat men niet met volle zekerheid van bijvoorbeeld Python zou kunnen zeggen. 
 
Rust legt ook een nadruk op typeveiligheid (typesafety) en geheugenveiligheid. Wanneer een taal of systeem geheugenveilig is, betekent dit dat het ontworpen is om toegangsfouten zoals bufferoverlopen, dangling pointers (verwijzingen naar vrijgegeven geheugen), en dubbele bevrijdingen van geheugen (double frees) te voorkomen. Deze soorten fouten kunnen leiden tot onvoorspelbaar gedrag, crashes, en beveiligingslekken zoals het uitvoeren van willekeurige code of informatie-lekken. Het is dus een goede eigenschap om te hebben.\autocite{Klabnik2022}

\subsection{Python}
Python was wellicht stukken trager in vergelijking met Rust, maar het blijft wel een taal waar iedereen gemakkelijk mee kan beginnen te werken. Voor het gemak van uitbreidbaarheid is dit dan weer een interessante optie om deze taal in optie te nemen. Zoals eerder vermeld door \textcite{TurnerTrauring2023} zou de performantie vooral bij de pipeline te merken zijn.
De echte vraag blijft momenteel nog altijd in hoeverre het minder performant zijn een grote zorg zal worden voor deze proof of concept.

\subsection{JavaScript}
Een zeer gekende scripting taal dat cross-platform gebruikt kan worden. Hoewel het eerder gericht is voor het gebruik in websites of GUI gerichte scripts, is het ook mogelijk om er cli-tools mee te maken. De performantie werd onderzocht. JavaScript is ook een dynamisch typerende taal, wat in theorie voor problemen zou kunnen zorgen in sommige use-cases, al lijkt dit initiëel geen zorg te zijn voor deze proof of concept.\autocite{MDN2024}

\section{Opbouw linter}
Aan het begin van dit project werd de beslissing genomen om direct met de ontwikkeling van een prototype aan de slag te gaan, wat leidde tot de zoektocht naar geschikte bronnen. In dit proces werden de artikelen van \textcite{BorgesLate2021}, die een stapsgewijze handleiding bieden voor het bouwen van een JavaScript-linter, als bijzonder nuttig beschouwd. Ondanks dat de lectuur van deze serie aanvankelijk de indruk wekte dat het project complexer zou zijn dan verwacht, veranderde dit vermoeden na verder onderzoek naar de structuur van .bib-bestanden. Het werd duidelijk dat de ontwikkeling van een linter voor BibLaTeX minder complex is dan voor programmeertalen. Dit is te wijten aan het feit dat programmeertalen loops, conditionele statements en een scala aan andere complexe elementen bevatten, terwijl BibLaTeX gekenmerkt wordt door een consistente en gestandaardiseerde structuur, waardoor de noodzaak voor het opleggen van regels aanzienlijk vermindert. Het was echter wel zeer nuttig om alle componenten te zien die aan bod kunnen komen bij het bouwen van een linter en om deze in gedachte te houden voor de effectieve uitwerking van de BibLaTeX-linter proof of concept.