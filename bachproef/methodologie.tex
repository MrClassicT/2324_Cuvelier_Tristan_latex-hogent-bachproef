%%=============================================================================
%% Methodologie
%%=============================================================================

\chapter{\IfLanguageName{dutch}{Methodologie}{Methodology}}%
\label{ch:methodologie}

%% TODO: In dit hoofstuk geef je een korte toelichting over hoe je te werk bent
%% gegaan. Verdeel je onderzoek in grote fasen, en licht in elke fase toe wat
%% de doelstelling was, welke deliverables daar uit gekomen zijn, en welke
%% onderzoeksmethoden je daarbij toegepast hebt. Verantwoord waarom je
%% op deze manier te werk gegaan bent.
%% 
%% Voorbeelden van zulke fasen zijn: literatuurstudie, opstellen van een
%% requirements-analyse, opstellen long-list (bij vergelijkende studie),
%% selectie van geschikte tools (bij vergelijkende studie, "short-list"),
%% opzetten testopstelling/PoC, uitvoeren testen en verzamelen
%% van resultaten, analyse van resultaten, ...
%%
%% !!!!! LET OP !!!!!
%%
%% Het is uitdrukkelijk NIET de bedoeling dat je het grootste deel van de corpus
%% van je bachelorproef in dit hoofstuk verwerkt! Dit hoofdstuk is eerder een
%% kort overzicht van je plan van aanpak.
%%
%% Maak voor elke fase (behalve het literatuuronderzoek) een NIEUW HOOFDSTUK aan
%% en geef het een gepaste titel.

Deze proof of concept werd opgesplitst in meerdere fasen. Voor het onderzoeken en effectief opstellen van hiervan waren enkel toegang tot een computer, technisch inzicht en logisch redeneren vereist. Het doel was om een open-source proof of concept te bekomen die kon dienen als basis om een effectieve volledig functionele linter op verder uit te bouwen.

Voor het gehele proces werden 14 weken ter beschikking gesteld. Het was de bedoeling om de fasen te verdelen over deze weken en om alsnog een week of twee over te houden ter 
reserve voor moesten er fouten bij de schattingen ingeslopen zijn.
De inschatting van de benodigde tijd werd bepaald op basis van het aantal onderzoek dat er moest gebeuren en van het verwachte resultaat aan het einde van deze fase. 

\section{Fase 1: Literatuurstudie}
Hier werd er onderzoek gedaan naar reeds bestaande linters, al dan niet gerelateerd aan LaTeX. Er werd gekeken naar de programmeertalen waarin deze gemaakt waren, hun werking, specificaties en functionaliteiten. Daarnaast mocht er ook niet vergeten worden om te kijken hoe een CI-pipeline effectief in zijn werk ging, zodat het mogelijk was om de linter op een efficiënte manier te integreren in workflows. Ook het verschil tussen BibLaTeX en BibTex werd onderzocht om te zien waar de oorzaak lag waardoor ze niet compatibel waren met elkaar.
Indien er andere relevante zaken opdoken die belang konden hebben aan het onderzoek, werden deze ook opgenomen om te onderzoeken. Op deze manier werd er geprobeerd een zo volledig mogelijk onderzoek te voeren. 
Een kijk naar hoe een linter \emph{echt} werkte. Deze kennis werd gebruikt als inspiratiebron voor het uiteindelijke proof of concept dat opgesteld werd in een latere fase. 
Hoewel de focus op deze fase het grootst was gedurende de aanvang van de onderzoeksperiode, was het een fase die parallel bleef doorlopen gezien er altijd extra info nodig kon zijn doorheen de andere fasen.
Het resultaat van deze fase was een (informeel) document waarin alle ondervindingen, onderzoeken en handige informatiebronnen bijgehouden werden over reeds bestaande linters en andere zaken die van pas kwamen tijdens zowel het onderzoeken als het ontwikkelen van deze proof of concept. Het document was een bron van kennis die gebruikt kon worden in verdere fasen.

%Fase 1 bestond zich dus uit de literatuurstudie, hierbij werd er gewerkt aan een informeel document dat zelf bijgehouden werd met allerhande info die van toepassing kon zijn tijdens het uitwerken van deze proef. Het is een fase die uiteindelijk doorheen alle weken heen actief was omdat er voortdurent handige info vergaard werd.

\section{Fase 2: Technische Analyse en Experimenten}
In deze fase was het doel om de lijst van gewenste functionele en niet-functionele requirements aan te vullen en deze te structureren naargelang hun prioriteit. Onder andere de keuze van de programmeertaal, eventuele libraries en andere softwaretools die gebruikt konden en zouden worden voor het maken en opzetten van de proof of concept werden hier gemaakt. De keuze van de programmeertaal werd bereikt door het opstellen van kleine prototypes in elke kandidaat-programmeertaal. Deze kregen werden dan getimed en daarnaast werden ook andere voor- en nadelen van elk bekeken om zo tot de beste optie te komen.


\section{Fase 3: Software Ontwikkeling (PoC - Proof of Concept)}
In deze fase gebeurde het uitdagende werk. De proof of concept werd ontwikkeld op basis van de verkregen bronnen. Een werkende versie werd opgesteld, inclusief testen om de werking te garanderen en om het eenvoudig uitbreidbaar te maken. Niettemin werd er ook documentatie geschreven, zodat alles wat er gebeurde duidelijk was voor elke vrijwilliger die een bijdrage wenste te leveren. Hoewel dit misschien al in fase 2 had mogen plaatsvinden, werd er ook een kanbanbord opgesteld om net iets meer overzicht te bewaren in de vooruitgang van de proof of concept en zodat er een duidelijk zicht was op de haalbaarheid van de MVP.

Aan het einde van deze fase stond er dus een werkende proof of concept beschikbaar op een GitHub-repository van de auteur van dit onderzoek, Tristan Cuvelier. 
Er werd verwacht dat deze fase wel enige tijd in beslag zou nemen gezien het verwachte resultaat en met een marge die ruim genoeg was voor problemen die opdoken tijdens het ontwikkelen van de software.

\section{Conclusie}
Wat zijn de requirements die in de POC zijn voldaan en welke dienen nog verricht te worden? TODO.