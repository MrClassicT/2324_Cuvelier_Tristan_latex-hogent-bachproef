%%=============================================================================
%% Methodologie
%%=============================================================================
%% TODO: In dit hoofstuk geef je een korte toelichting over hoe je te werk bent
%% gegaan. Verdeel je onderzoek in grote fasen, en licht in elke fase toe wat
%% de doelstelling was, welke deliverables daar uit gekomen zijn, en welke
%% onderzoeksmethoden je daarbij toegepast hebt. Verantwoord waarom je
%% op deze manier te werk gegaan bent.
%% 
%% Voorbeelden van zulke fasen zijn: literatuurstudie, opstellen van een
%% requirements-analyse, opstellen long-list (bij vergelijkende studie),
%% selectie van geschikte tools (bij vergelijkende studie, "short-list"),
%% opzetten testopstelling/PoC, uitvoeren testen en verzamelen
%% van resultaten, analyse van resultaten, ...
%%
%% !!!!! LET OP !!!!!
%%
%% Het is uitdrukkelijk NIET de bedoeling dat je het grootste deel van de corpus
%% van je bachelorproef in dit hoofstuk verwerkt! Dit hoofdstuk is eerder een
%% kort overzicht van je plan van aanpak.
%%
%% Maak voor elke fase (behalve het literatuuronderzoek) een NIEUW HOOFDSTUK aan
%% en geef het een gepaste titel.

\chapter{\IfLanguageName{dutch}{Methodologie}{Methodology}}%
\label{ch:methodologie}

Deze \acrlong{POC} is opgesplitst in meerdere fasen. Voor het onderzoeken en effectief opstellen hiervan waren alleen toegang tot een computer, technisch inzicht en logisch redeneren vereist. Het doel was om een open-source \acrlong{POC} te ontwikkelen die als basis kon dienen voor een volledig functionele linter.

Voor het gehele proces waren 14 weken beschikbaar. De fasen zijn verdeeld over deze periode met een buffer van twee weken om eventuele fouten in de schattingen op te vangen. De benodigde tijd werd ingeschat op basis van het aantal vereiste onderzoeken en het verwachte resultaat aan het einde van elke fase.

\section{Fase 1: Literatuurstudie}
In deze fase werd onderzoek gedaan naar bestaande linters, met en zonder betrekking tot \LaTeX{}. Er werd gekeken naar de programmeertalen waarin deze linters waren geschreven, hun werking, specificaties en functionaliteiten. Daarnaast werd onderzocht hoe een \acrshort{CI}-pipeline effectief werkt om de linter efficiënt te kunnen integreren in workflows. Ook werd het verschil tussen BibLaTeX en BibTex onderzocht om de oorzaak van hun incompatibiliteit te achterhalen. Eventuele andere relevante zaken die tijdens het onderzoek naar voren kwamen, werden ook meegenomen.

Het resultaat van deze fase was een informeel document met alle bevindingen, onderzoeken en nuttige informatiebronnen over bestaande linters en andere relevante onderwerpen. Dit document diende als kennisbron voor de volgende fasen en werd uiteindelijk verwerkt in de literatuurstudie.

\section{Fase 2: Technische Analyse en Experimenten}
In deze fase was het doel om de lijst van functionele en niet-functionele requirements aan te vullen en te prioriteren. De keuze van de programmeertaal, libraries en andere softwaretools werd hier gemaakt. Deze keuzes werden gebaseerd op experimenten met kleine prototypes in elke kandidaat-programmeertaal, waarbij zowel de snelheid als andere voor- en nadelen werden geëvalueerd.

\section{Fase 3: Software Ontwikkeling | \acrlong{POC}}
In deze fase werd de \acrlong{POC} ontwikkeld op basis van de verzamelde bronnen. Een werkende versie werd opgesteld, inclusief testen om de werking te garanderen en de uitbreidbaarheid te vergemakkelijken. Er werd ook documentatie geschreven om duidelijkheid te verschaffen aan iedereen die bijdroeg aan het project. Een kanbanbord werd opgezet om het overzicht over de voortgang te behouden en de haalbaarheid van de MVP te bewaken.

Aan het einde van deze fase was een werkende \acrlong{POC} beschikbaar op een GitHub-repository\footnote{\url{https://github.com/MrClassicT/bibla}} van de auteur van dit onderzoek, Tristan Cuvelier. Deze fase nam aanzienlijke tijd in beslag vanwege het verwachte resultaat, met een ruime marge voor eventuele problemen tijdens de ontwikkeling.

\section{Conclusie}

Tot slot wordt er een conclusie getrokken waarin de mate van succes van deze \acrlong{POC} wordt bepaald. De opgeleverde functionaliteiten worden geëvalueerd, evenals de mate waarin aan de niet-functionele requirements is voldaan. Daarnaast worden de resterende werkpunten en de toekomstvisie van deze \acrlong{POC} besproken.