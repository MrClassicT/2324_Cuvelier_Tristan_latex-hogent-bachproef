\chapter{Uitwerkingsfase}
\label{ch:uitwerkingsfase}
%In deze fase gebeurde het uitdagende werk. De proof of concept werd ontwikkeld op basis van de verkregen bronnen. Een werkende versie werd opgesteld, inclusief testen om de werking te garanderen en om het eenvoudig uitbreidbaar te maken. Niettemin werd er ook documentatie geschreven, zodat alles wat er gebeurde duidelijk was voor elke vrijwilliger die een bijdrage wenste te leveren. Hoewel dit misschien al in fase 2 had mogen plaatsvinden, werd er ook een kanbanbord opgesteld om net iets meer overzicht te bewaren in de vooruitgang van de proof of concept en zodat er een duidelijk zicht was op de haalbaarheid van de MVP.

%Aan het einde van deze fase stond er dus een werkende proof of concept beschikbaar op een GitHub-repository van de auteur van dit onderzoek, Tristan Cuvelier. 
%Er werd verwacht dat deze fase wel enige tijd in beslag zou nemen gezien het verwachte resultaat en met een marge die ruim genoeg was voor problemen die opdoken tijdens het ontwikkelen van de software.

Deze fase vond iets later plaats dan initiëel verwacht, maar verliep ook vlotter dan verwacht. Dat allemaal dankzij het ontdekken van bibl. Een gepaste \emph{rebrand} naam vinden voor bibl was misschien geen prioriteit, maar er werd toch besloten er eentje te vinden. Na wat brainstormen werd de naam bibla bedacht. Waar bibl wellicht stond voor bibTeX-Linter of iets soortgelijk, zal bibla simpelweg refereren naar BibLaTeX. Gezien Biblal minder leuk klonk, werd er gekozen voor bibla.

Nu deze lastige zaak afgehandeld is, werd er eerst een hele rebrand uitgevoerd op de fork van bibl; bibla was geboren.
\\
\\
In het begin was het zeer lastig om eigen regels toe te voegen, dit omdat het ondanks de grondige analyse die erop plaats vond, nog altijd een \emph{vreemd} project was. Daarom werd er besloten om eerst bestaande regels te controleren en deze te vergelijken met de eigen lijst van functionele requirements.
Zo konden er enerzijds taken worden geschrapt van zaken die reeds geïmplementeerd waren en kon er een overzicht gemaakt worden van de regels die aangepast diende te worden. 

Op deze manier werd er toch al nuttige vooruitgang geboekt en schepte dit een vertrouwensband met de bestaande codebase. Eens enkele waren regels aangepast, werden er ook eigen regels opgesteld die nodig waren.
% ----
\section{bibla - 2.0.5}
\label{sec:bibla205}
Gezien \texttt{bibl} de voorganger is van \texttt{bibla}, werd er besloten om \texttt{bibla} vanaf versie 2.0.0 te laten starten. Dit met de insteek dat de grote basis van \texttt{bibl} zeker niet vergeten mag worden. Naast het uitschrijven van deze proef, werd er ook nog rustig verdergewerkt aan \texttt{bibla} zelf, waardoor sommige bugs die in deze codeblocks nog te zien zijn, mogelijks al een verandering ondergaan zijn in latere versies. Onder de GitHub-issues\footnote{\url{https://github.com/MrClassicT/bibla/issues}} kunnen de gekende bugs geraadpleegt worden en kunnen er ook nieuwe bugs gemeld worden door onder andere de gebruikers. Gezien code altijd blijft evolueren, was het niet mogelijk om altijd de meest recente versie van de code te gebruiken. Dit is waarom er expliciet vermeld wordt dat er in deze proef gekeken wordt naar versie 2.0.5 omdat veranderingen zullen blijven gebeuren; wat ook het doel is van deze \acrshort{POC}.

\subsection{Behouden regels}
Deze sectie bevat de regels die volledig doorgekomen zijn vanuit \texttt{bibl} en dus geen veranderingen ondergaan zijn. Wèl is het mogelijk dat hier regels bij vermeld staan waarvan er veranderingen zijn doordat het configuratiebestand veranderingen ondergaan heeft.

Zo zijn regels voor wat de \texttt{Entries} (E) betreffen: E01 (voornaam auteur mag niet worden afgekort), E02 (middennamen die al dan niet mogen worden afgekort), E03 (vermijd het gebruik van 'Et Al' in auteursveld), E04 (bestanden moeten relatief pad gebruiken), E05 (gelinkte bestanden die afwezig zijn), E06 (onjuist DOI-formaat) en E07 (onjuist ISBN-formaat) gelijk gebleven. 

De \texttt{Database} (D) regels: D00 (Alfabetische volgorde van de bronnen), D01 (Preambule is de start van het document), D02 (Mogelijke duplicaten op basis van titel) zijn ook ongewijzigd gebleven.

De \texttt{Missing} (M) regels: M00 (Geen auteurs of redacteurs), M01 (Missend verplicht veld), M02 (Missend optioneel veld) zijn ook ongewijzigd gebleven wat de code betreft. Echter is hier wel de configuratie grondig aangepakt geweest voor M01 en M02, gezien HOGENT\footnote{\url{www.hogent.be}} zelf eigen voorkeuren heeft over welke bibliografische gegevens van een bron bijgehouden dienen te worden. Ook voor de \texttt{Unrecognized} (U) regels, zijn er enkel wijzigingen gebeurt in de configuratie. 

De \texttt{Unrecognized} (U) regels zijn ook totaal onaangepast gebleven. Hetzelfde kan gezegd worden voor de \texttt{Textual} (T) regels; opnieuw met uitzondering voor T01 (aantal spaties die als indentatie gebruikt dienen te worden), waar de configuratie van werd aangepast; uiteraard met uitzondering voor regel T00 (enkel ASCII karakters) die weggehaald werd gezien deze niet meer relevant is.

\subsection{Aangepaste regels}
\paragraph{E00 - Sleutelformaat}
\label{rule:E00}
Eén van de regels die aangepast diende te worden, was regel E00. Deze regel controleert het formaat van entry keys. Dit is vooral belangrijk om de consistentie te garanderen en om ervoor te zorgen dat er niet plots door inconsistenties een verwijzing niet gevonden kan worden. De regel ziet er als volgt uit:

\begin{minted}[autogobble, breaklines, linenos]{python3}
@register_entry_rule('E00', 'Keys of published works should have format `AuthorYEARa`')
def key_format(key, entry, database):
    """Raise a linter warning when entry key is not of format `AuthorYEARa`.

    E.g. an entry with values
    ```
    author = {Arthur B Cummings and David Eftekhary and Frank G House},
    date   = {2003-02-02}
    ```
    should have key `Cummings2003`.
    If another entry with the same year and main author is present,
    their keys should have formats `Cummings2003a` and `Cummings2003b`.
    This rule only applies when `date` and at least one author are set.

    :param key: The key of the current bibliography entry
    :param entry: The current bibliography entry
    :param database: All bibliography entries
    :return: True if the current entry's key has the specified format or
    year or author are not specified, False otherwise.
    """
    if 'date' not in entry.fields or len(entry.persons.get('author', [])) == 0:
        return True
    try:
        author = entry.persons['author'][0]
    except KeyError:
        try:
            author = entry.persons['editor'][0]
        except KeyError:
            return True
    names = author.rich_prelast_names + [name for last_name in author.rich_last_names for name in last_name.split('-')]
    date = entry.fields['date']
    year = re.search(r'\d{4}', date).group()

    # Check for 'EtAl' in the key when there are more than 3 authors
    if len(entry.persons.get('author', [])) >= 3 and 'EtAl' in key.lower():
        correct_key_unicode = names[0] + 'EtAl' + year
    # Check for two names in the key when there are exactly 2 authors
    elif len(entry.persons.get('author', [])) == 2:
        correct_key_unicode = "".join([str(name) for name in names[:2]]) + year
    else:
        correct_key_unicode = "".join([str(name) for name in names]) + year

    correct_key_ascii = unidecode(str(correct_key_unicode))
    regex = re.compile(correct_key_ascii + r'[a-zA-Z]?')
    return bool(regex.match(key))
\end{minted}
\captionof{listing}[bibla: E00 - Correct \texttt{key} formaat]{bibla: E00 - Sleutels moeten het juiste formaat hebben. De aangepaste versie. Aanpassingen waren nodig vanwege het gebruik van het 'date' field in plaats van het voormalige 'year' field. \label{lst:bibla_AR_E00}}


Gezien bij bibTeX de datum opgesplitst werd in \texttt{year}, \texttt{month} en \texttt{day} velden, diende deze regel aangepast te worden om rekening te houden met het \texttt{date} veld in de plaats (zie regel 21). Alsook diende er een \acrfull{REGEX} toegevoegd te worden om rekening te houden met het veranderde datum formaat. Het jaar moest immers uit het date field gehaald worden gezien het geen veld op zich meer was, zie regel 32. Daarnaast werd er ook een controle toegevoegd voor wanneer er geen author aanwezig was en bijvoorbeeld enkel een editor. \texttt{bibl} hield er namelijk geen rekening mee dat er enkel editors aanwezig zouden kunnen zijn in een bepaalde bron. Bij het uitwerken van deze proef is er een bron in de bibliografie geraakt waarbij dit het geval was. Dit had als gevolg dat de linter crashte bij het linten van dit bibliografisch bestand.
Daarnaast moest het ook mogelijk zijn om achternamen van twee auteurs te gebruiken in een sleutel of zelf 'EtAl' indien het er drie of meer waren. Ook hiervoor werden er enkele controles op punt gezet om hier rekening mee te houden, zie code-regel 35 tot en met 41.

\paragraph{E08 - \texttt{pages} veld formaat}
Deze regel leek niet te werken vanwege het veld waarop gecontroleerd werd. Er werd naar het \texttt{page} veld gezocht, maar het veld waar effectief naar gezocht diende te worden staat gekend als \texttt{pages} in BibLaTeX. Uiteindelijk werd deze regel een samenvoeging van de voormalige regel E08 (pagina formaat moet xx--yy zijn) en E09 (eindpagina dient groter te zijn dan de startpagina) uit \texttt{bibl}.

\paragraph{E09 - Correct \texttt{date} formaat}
Zoals bij regel E00 (\ref{rule:E00}) reeds besproken was, is het gebruik van data wat veranderd. Nu er geen apart veld meer gebruikt wordt voor \texttt{year}, \texttt{month} en \texttt{day}, dient er een extra controle te gebeuren om te zien of de datum van een bron wel goed geformatteerd is. Waar de originele regel van bibl enkel keek of de maand weldegelijk in 'MMM' formaat was, dient er nu gekeken te worden dat het formaat gelijk is aan 'YYYY-MM-DD'. Deze regel vervangt de in \texttt{bibl} gekende regel E10 waarbij de maand in \texttt{mmm}-formaat diende te zijn.

\begin{minted}[autogobble, breaklines, linenos]{python3}
    if 'date' not in entry.fields:
        return True
    date = entry.fields['date']
    regex = re.compile(r'^(\d{4})-(\d{2})-(\d{2})$')
    match = regex.match(date)
    if not match:
        regex = re.compile(r'^(\d{4})-(\d{2})$')
        match = regex.match(date)
        if not match:
            regex = re.compile(r'^(\d{4})$')
            match = regex.match(date)
            if not match:
                return False
            return True
        year, month = map(int, match.groups())
        if not (1 <= month <= 12):
            return False
        return True
    year, month, day = map(int, match.groups())
    if not (1 <= month <= 12):
        return False
    if not (1 <= day <= 31):
        return False
    return True
\end{minted}
\captionof{listing}[bibla: E09 - Correct \texttt{date} formaat]{bibla: E09 - Correct \texttt{date} formaat. Deze regel werd gewijzigd gezien het vermelden van data gewijzigd is. Gebruikt nu \texttt{date} veld in plaats van de \texttt{year}, \texttt{month} en \texttt{day} velden. Dit is de logica. \label{lst:bibla_AR_E09}}

Ook hier is er op meerdere plaatsen weer een \acrfull{REGEX} te zien. \acrshort{REGEX} zijn krachtig voor het controleren van bepaalde vaste tekstuele structuren. In deze use case zijn ze bijvoorbeeld zeer handig om het \texttt{date} veld op te splitsen. Bij de eerste check, zie regel 4, kan al direct gezien worden of het een '4-2-2' formaat is, of toch eerder een '2-2-4' formaat. Gezien zowel de maand als dag met twee getallen voorgesteld worden, is het belangrijk om hier een minimale vorm van controle op uit te voeren. Zo kan het maandnummer nooit groter zijn dan 12 en zijn er maar maximaal 31 dagen tijdens de langste dagen. Hoewel dit niet elke fout kan tegengaan, is het wel handig om deze minimale controles uit te voeren om zoveel mogelijk garantie te bieden dat er geen grove fouten optreden op vlak van consistentie in de \texttt{date} velden. Regels 6 tot en met 18 voeren de controle uit om te zien wanneer de dag en/of maand moest ontbreken, om te zien dat ook dan het formaat zo goed mogelijk gerespecteerd wordt.


\subsection{Nieuwe regels}
\paragraph{E1O - Gebruik van Velden}
\label{rule:E10-field-preferences}
Zoals bij de aanpassingen gemerkt kon worden, was het gebruik van andere velden toch wel een aanpassing. Echter is het niet te voorkomen dat er nooit nog de \emph{oudere} velden gebruikt worden om data in op te slaan. Om de gebruikers zo goed mogelijk te ondersteunen, werd er een regel opgesteld waardoor de voorkeursvelden en hun aliassen konden gecontroleerd worden. De regel gaat op zoek naar oudere varianten/naamgevingen en suggereerd dan het nieuwere veld aan in de plaats. 

Er kan terug gedacht worden aan de \texttt{year}, \texttt{month} en \texttt{day} velden, waarvoor nu de voorkeur gaat om \texttt{date} te gebruiken.

De regel die dit controleert zag er als volgt uit:

\begin{minted}[autogobble, breaklines, linenos]{python3}
        def register_variant_rule(entry_type, field, variant):
        rule_id = 'E10{}{}'.format(entry_type.capitalize(), field.capitalize())
        message = "Entry type {} - use {} instead of {}!".format(entry_type, variant, field)
        
        @register_entry_rule(rule_id, message)
        def check_variant_field(key, entry, database, entry_type=entry_type, field=field, variant=variant):
        """Raise a linter warning when a specific field is used instead of its variant.
        
        This function checks if the variant field is required and if the specific field is present.
        If these conditions are met, it suggests to use the variant field instead of the specific field.
        
        :param key: The key of the current bibliography entry
        :param entry: The current bibliography entry
        :param database: All bibliography entries
        :param entry_type: Anchor variable to pass the local variable `entry_type` from outer scope
        :param field: The field that is being checked
        :param variant: The field that should be used instead
        :return: False if the specific field is used instead of its variant, True otherwise
        """
        if entry.type == entry_type and field in entry.fields:
            return False
        else:
            return True
\end{minted}
\captionof{listing}[bibla: E10 - Gebruik aanbevolen velden]{bibla: Nieuwe Regel | E10 - Gebruik aanbevolen velden in plaats van gelijkaardigen. \label{lst:bibla_NR_E10}}

Net zoals bij de originele regels van bibl, wordt er ook bij de nieuwe aandacht besteed aan een correcte documentatie van de code. Het is namelijk dankzij deze zorgvuldige documentatie dat gebruikers en bijdragers (Engels: Contributors) in staat zijn om te begrijpen wat er gaande is en hun bijdrage te leveren zonder onnodig tijd te moeten verspillen aan het ontcijferen van bestaande code.
    
Bij deze regel kan ook duidelijk gezien worden dat er inspiratie gehaald werd bij een reeds bestaande regels die op een zeer gelijkaardige manier te werk gaan. De regels M01 en M02 zijn een grote inspiratie geweest. De regels zijn simpel maar zeer efficiënt. Er wordt een lijst \emph{dictionary} bijgehouden waarbij de 'oude' velden als \emph{key} dienen en de 'nieuwe' als \emph{value}.
% ---

\paragraph{E11 - Gebruik van Aliassen}
\label{rule:E11-alias-preferences}
Binnen BibLaTeX zijn er vaak vele varianten beschikbaar voor bepaalde types bronnen, deze varianten vereisen dan dezelfde fields als het \emph{originele} type. Om consistentie te garanderen, gaat de voorkeur uit om altijd het \emph{originele} type te gebruiken in plaats van één van de aliassen. Deze methode werd op een gelijkaardige manier uitgewerkt als E10, M01 en M02 waar er gekeken wordt naar de bepaalde velden die al dan niet gebruikt worden. Het enige verschil is dat er hier niet naar de velden gekeken worden van een \emph{entry} maar wel naar het type van de \emph{entry}.
%---
\paragraph{E12 - Geen startpagina's in \texttt{URL} veld}
Startpagina's zijn te algemeen en verwijzen meestal niet naar een geldige bron, daarom is het van groot belang dat de correcte URL opgeslagen wordt, zodat ook na het raadplegen van een bron voor de eerste keer, dezelfde bron eenvoudig achterhaald kan worden. Ook voor in de bibliografie is het van groot belang dat dit correct opgeslagen wordt. Deze regel wordt nagekeken door opnieuw gebruik te maken van \acrshort{REGEX}. De code binnen deze regel ziet er als volgt uit:
\begin{minted}[autogobble, breaklines, linenos]{python3}
    if 'url' not in entry.fields:
        return True
    url = entry.fields['url']
    regex = re.compile(r'^(https?://)?[^/]+/?$')
    if regex.match(url):
        return False
    return True
\end{minted}
\captionof{listing}[bibla: E12 - Geen startpagina's gebruiken]{bibla: Nieuwe Regel | E12 - Gebruik geen startpagina's in het \texttt{URL} veld, logica. \label{lst:bibla_NR_E12}}

Op regel 4, in de \acrshort{REGEX} kan er gezien worden dat er gekeken wordt naar hoe een typische URL eruit kan zien, rekeninghoudende met al dan niet optionele componenten uit een URL.
% ---

\paragraph{E13 - Enkel kritische pad in \texttt{URL} veld}
Naast dat er niet enkel startpagina's gebruikt mogen worden, is het ook ongewenst dat de URL onnodig lang is. Om deze voorvallen te voorkomen, wordt er een check gedaan op enkele karakters die typisch voorkomen in \emph{te lange} URLS. Hieronder worden bijvoorbeeld URL's verstaan waarin er een directe verwijzing gebeurt naar een specifieke zin uit een webpagina. Het wordt met behulp van een opsomming van deze karakters gecontroleerd zoals hieronder te zien is.
\begin{minted}[autogobble, linenos, breaklines]{python3}
if ('#' or '?' or '%') in url:
\end{minted}
\captionof{listing}[bibla: E13 - Enkel kritische pad van URL]{bibla: Nieuwe Regel | E13 - Gebruik enkel het kritische pad in het \texttt{URL} veld, logica. \label{lst:bibla_NR_E13}}


%---
\paragraph{M03 - Speciale karakters}

Deze regel zal mogelijks later van naam veranderen, gezien het niet direct onder de meest intuïtieve categorie staat (M). 'M' wordt doorgaans gebruikt voor 'Missing' of ontbrekende zaken aan te duiden. Al kan er gediscussiëerd worden of dit dan niet net toch de juiste categorie zou zijn, gezien het om een ontbrekend karakter gaat. Speciale karakters dienen namelijk vervangen te worden door het commando om ze te genereren. Het commando is doorgaans gewoon een \backslash voor het karakter plaatsen. Dit zorgt ervoor dat LaTeX het speciale karakter niet gaat zien als een commando dat het moet uitvoeren. 


\begin{minted}[autogobble, breaklines, linenos]{python3}
    @register_entry_rule('M03', 'Special characters should be replaced by the command to generate them: %, &, $, #, _, \\, ~, ^, |')
    def check_special_characters(key, entry, database):
        """Raise a linter warning when a field contains special characters that should be replaced.
        :param key: The key of the current bibliography entry
        :param entry: The current bibliography entry
        :param database: All bibliography entries
        :return: True if no field contains special characters, False otherwise.
        """
        special_chars = ['%', '&', '$', '#', '_', '\\', '~', '^', '|']
    
        for field in entry.fields.values():
            if any(char in field for char in special_chars):
                for char in special_chars:
                    if char in field and field[field.index(char) - 1] != '\\':
                        return False
        return True
\end{minted}
\captionof{listing}[bibla: M03 - Speciale karakters dienen gegenereerd te worden]{bibla: Nieuwe Regel | M03 - Speciale karakters dienen met commando gegenereerd te worden. \label{lst:bibla_NR_M03}}

Hoewel deze versie gebruikt wordt in versie 2.0.5 van bibla, zit hier nog een kleine \emph{bug} in. Het is namelijk zo dat deze regel elk veld zal controleren op speciale karakters. Hoewel dit over het algemeen goed is, gooit het hierdoor ook foutieve waarschuwingen op het \texttt{url} veld.


%----

\subsection{Diverse aanpassingen}
Ondanks dat \texttt{bibl} een uitstekende start was voor deze \acrlong{POC}, waren er nog een paar zaken ter verbetering vatbaar. Foutafhandelingen waren hier een deel van. Ze dienen niet direct als een regel gezien te worden, maar ze kunnen wel voorkomen door regels die overtreden worden. Zo was bijvoorbeeld het bevatten van duplicate entry keys een reden waardoor de applicatie het soms begaf. Dit werd opgelost door een 'exception handler', ook wel foutafhandelaar genoemd, toe te voegen. Hoewel dan niet het hele bestand gelind werd, vanwege de vroegtijde fout in het programma, wordt er nu tenminste wel op een minder schokkerende wijze getoond wat er mis is en gewijzigd dient te worden. Hetzelfde werd voorzien voor wanneer er gewoon lege entries waren.  

\begin{minted}[autogobble, breaklines, linenos]{python3}
try:
    bib_data = parse_file(bibliography, macros=MONTH_NAMES)
except BibliographyDataError as e:
    # Extract the key from the error message
    match = re.search(r'repeated bibliograhpy entry: (.*)', str(e))
    if match:
        duplicate_key = match.group(1)
        print(f"{bibliography} D03: Duplicate entry with key '{duplicate_key}'")
    else:
        print(f"Warning: {e}")
    return []
except pybtex.scanner.TokenRequired as e:
    print(f"E00: {e}")
    return []
except Exception as e:
    print(f"Error: {e}")
    return []
\end{minted}
\captionof{listing}[bibla: Extra foutafhandelingen]{bibla: Diverse aanpassing | Foutafhandelingen om de tool properder te laten werken. \label{lst:bibla_foutafhandelingen_voorbeeld}}

In dit codeblock, genomen uit \texttt{lint.py}, kan er gezien worden dat er aan extra foutafhandelingen gedaan wordt. De regels die hier overtreden worden, staan op dit ogenblik vast gecodeerd en zijn minder dynamisch dan doorheen de rest van de applicatie gewend is, maar ze zijn afgestemd geweest en als de testen als betrouwbaar gezien kunnen worden, zou dit kloppen met wat de oorzaak achterliggend effectief is. Dit was de meest eenvoudige en duidelijke manier om de gebruiker te laten weten wat er mis is zonder kostbare tijd te verliezen voor het verder ontwikkelen van andere delen van de applicatie.

\subsection{Configuratie}
De standaardconfiguratie van bibl was een zeer goede basis om van verder te gaan voor bibla. Maar zoals al vaker aangegeven, zijn er verschillen tussen bibTeX en bibLaTeX die ervoor zorgen dat er aanpassingen dienen te gebeuren. Op de configuratie van bibl\footnote{\url{https://gitlab.com/arnevdk/bibl/-/blob/master/bibl/bibl.yml}} zijn er zaken veranderd, maar ook extra toegevoegd. Zoals voor bijvoorbeeld regel E10 (\ref{rule:E10-field-preferences}) waar er gekeken werd naar bepaalde velden waarbij er een voorkeur is om er een synoniem van te gebruiken. Waarvan hieronder het stukje configuratie in YAML-formaat.


\begin{minted}[autogobble, breaklines, linenos]{yaml}
alternate_fields:
  # preferred: [other]
  date: [year] # Month and day will not be used alone, so when we just check for the year, that'll be fine.
  journaltitle: [journal]
  institution: [school]
\end{minted}
\captionof{listing}[bibla: Configuratie | E10 - Gebruik aanbevolen velden]{bibla: Configuratie | E10 - Gebruik aanbevolen velden in plaats van gelijkaardigen. \label{lst:bibla_AC_E10}}
   
De overige configuratie kan geraadpleegt worden op de \texttt{bibla} repository in \texttt{bibla.yml}\footnote{\url{https://github.com/MrClassicT/bibla/blob/master/bibla/bibla.yml}}, wat de algemene configuratie voor \texttt{bibla} bevat.

\subsection{Pipelines}
\texttt{bibl} werkte met GitLab, \texttt{bibla} werkt met GitHub. Het is daarom niet onlogisch dat er nog enkele andere zaken aangepast dienen te worden om alles vlot te laten verlopen. Hierbij denken we vooral aan de pipelines die gebruikt worden. \texttt{bibl} bevat slechts één pipeline bestand\footnote{\url{https://gitlab.com/arnevdk/bibl/-/blob/master/.gitlab-ci.yml}}, een continuous integration. Al gebeurt er in deze pipeline ook een \emph{deploy} van een statische website die alle linterregels bevat.

Op GitHub worden pipelines gekend onder \emph{workflows}, wat reflecteert naar wat een pipeline effectief inhoudt. Gezien de \emph{workflows} iets anders zijn dan de gewone pipelines op GitLab, diende er enkele wijzigingen te gebeuren. Zo diende de pipelines te komen in een folder \texttt{.github/workflows} om herkend te worden. Hierin werd er een omgevormde CI-pipeline geplaatst, gebaseerd op de originele van \texttt{bibl}.

\paragraph{CI-pipeline}
\begin{minted}[autogobble, breaklines, linenos]{yaml}
name: CI Workflow

on:
  push:
    branches:
      - master

jobs:
  build:
    runs-on: ubuntu-latest
    strategy:
      matrix:
        python-version: ["3.12"]
    steps:
    - uses: actions/checkout@v4

    - name: Set up Python ${{ matrix.python-version }}
      uses: actions/setup-python@v4
      with:
        python-version: ${{ matrix.python-version }}

    - name: Install dependencies
      run: |
        python -m pip install --upgrade pip
        pip install git+https://${{secrets.BIBLA_GITHUB_TOKEN}}@github.com/mrclassict/bibla.git
        pip install --upgrade setuptools wheel
        pip install -e .[dev]
        pip install flake8

    - name: Static Analysis with Flake8
      run: |
        mkdir out
        flake8 --append-config .flake8 --ignore=E501 | tee out/lint_result.txt
        num_warnings=$(wc -l < out/lint_result.txt)
        echo "$num_warnings warnings!"
        echo "Generating badge..."
        anybadge -v=$num_warnings -f=out/flake8.svg -l=flake8 10=green 1000=orange 10000=red
        cat out/lint_result.txt
        if [ $num_warnings -gt 0 ]; then exit 1; fi

    - name: Upload artifacts
      uses: actions/upload-artifact@v4
      with:
        name: flake8-results
        path: |
          out/flake8.svg
          out/lint_result.txt
\end{minted}
\captionof{listing}[bibla: \acrshort{CI}-pipeline]{bibla: \acrshort{CI}-pipeline | \acrfull{CI} pipeline overzicht \label{lst:bibla_CI}}

Bij het analyseren van deze pipeline, kan er gezien worden dat er vier zaken gebeuren:
\begin{enumerate}
    \item De juiste Python-versie installeren (in dit voorbeeld 3.12)
    \item Dependencies installeren
    \item De code linten met flake8
    \item Artifacts uploaden
\end{enumerate}

Deze stappen gebeuren op een virtuele machine op de GitHub servers en ze gebeuren elke keer dat er een \emph{push} gebeurt naar de 'master' branch.
Hoewel in deze pipeline net gelijk bij \texttt{bibl} Flake8 gebruikt wordt om de code te linten, werd er ondertussen geleerd dat Ruff wellicht een interessantere optie is. Naar de toekomst toe kan het interessant zijn om hier eens naar te kijken en eventueel over te schakelen van linting-tool.


\paragraph{CD-pipeline}
Om ervoor te zorgen dat bij elke \emph{release} (of uitgave in het Nederlands) \texttt{bibla} eenvoudig en consistent naar de pakketmanager gestuurd kon worden, werd er ook een CD-pipeline voorzien. Het opzetten van deze pipeline ging dankzij de templates die GitHub voorziet in minder dan vijf minuten! Het enige dat zelf diende te gebeuren, was een \acrshort{API}-token (een vorm van zowel identificatie en authenticatie) voorzien en de CD-pipeline-template deed de rest.

\begin{minted}[autogobble, breaklines, linenos]{yaml}
# This workflow will upload a Python Package using Twine when a release is created
# For more information see: https://docs.github.com/en/actions/automating-builds-and-tests/building-and-testing-python#publishing-to-package-registries

# This workflow uses actions that are not certified by GitHub.
# They are provided by a third-party and are governed by
# separate terms of service, privacy policy, and support
# documentation.

name: Upload Python Package

on:
  release:
    types: [published]

permissions:
  contents: read

jobs:
  deploy:

    runs-on: ubuntu-latest

    steps:
    - uses: actions/checkout@v4

    - name: Set up Python
      uses: actions/setup-python@v3
      with:
        python-version: '3.x'
    - name: Install dependencies
      run: |
        python -m pip install --upgrade pip
        pip install build
        pip install markdown

    - name: Build package
      run: python -m build

    - name: Publish package
      uses: pypa/gh-action-pypi-publish@27b31702a0e7fc50959f5ad993c78deac1bdfc29
      with:
        user: __token__
        password: ${{ secrets.PYPI_API_TOKEN }}

\end{minted}
\captionof{listing}[bibla: \acrshort{CD}-pipeline]{bibla: \acrshort{CD}-pipeline | \acrfull{CD} pipeline overzicht \label{lst:bibla_CD}}

\subsection{Testen}
Wat de testen betreft bij deze \acrlong{POC}, werd er zich gebaseerd op de manier waarop \texttt{bibl} gebruikt wordt. Er werden test.bib-bestanden aangemaakt die elk voor een bepaalde regel verantwoordelijk dienden te zijn. Zo kon er tijdens het ontwikkelen getest worden of de regel al dan niet werkt zoals verwacht. Uiteindelijk bleek wel dat deze test.bib-bestanden wat vaker revisies mochten ondergaan om accuraat en representatief te blijven. \acrshort{AI} heeft hier ook deels geholpen om deze te genereren, maar gezien er wel relevante bronnen en input nodig waren, was het niet altijd de beste tactiek. Vooral omdat online heel veel BibTeX-bronnen staan en die dus ook vaker aanbevolen worden door \acrshort{AI}.

Testen werden manueel uitgevoerd door het \mintinline{python3}{bibla lint test.bib} commando uit te voeren.

